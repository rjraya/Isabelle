%-----------------------------------------------------------------------------------------------------
%	INCLUSIÓN DE PAQUETES BÁSICOS
%-----------------------------------------------------------------------------------------------------
\documentclass[a4paper,12pt]{article}
%-----------------------------------------------------------------------------------------------------
%	SELECCIÓN DEL LENGUAJE
%-----------------------------------------------------------------------------------------------------
% Paquetes para adaptar Látex al Español:

%-----------------------------------------------------------------------------------------------------
%	SELECCIÓN DE LA FUENTE
%-----------------------------------------------------------------------------------------------------
% Fuente utilizada.
\usepackage{courier}                    % Fuente Courier.
\usepackage{microtype}                  % Mejora la letra final de cara al lector.
%-----------------------------------------------------------------------------------------------------
%	ALGORITMOS
%-----------------------------------------------------------------------------------------------------
\usepackage{algpseudocode}
\usepackage{algorithmicx}
\usepackage{algorithm}
\usepackage{listings}

%-----------------------------------------------------------------------------------------------------
%	IMÁGENES
%-----------------------------------------------------------------------------------------------------
\usepackage{float}
\usepackage{placeins}
%-----------------------------------------------------------------------------------------------------
%	ESTILO DE PÁGINA
%-----------------------------------------------------------------------------------------------------
% Paquetes para el diseño de página:
\usepackage{fancyhdr}               % Utilizado para hacer títulos propios.
\usepackage{lastpage}               % Referencia a la última página. Utilizado para el pie de página.
\usepackage{extramarks}             % Marcas extras. Utilizado en pie de página y cabecera.
\usepackage[parfill]{parskip}       % Crea una nueva línea entre párrafos.
\usepackage{geometry}               % Asigna la "geometría" de las páginas.
% Se elige el estilo fancy y márgenes de 3 centímetros.
\pagestyle{fancy}
\geometry{left=3cm,right=3cm,top=3cm,bottom=3cm,headheight=1cm,headsep=0.5cm} % Márgenes y cabecera.
% Se limpia la cabecera y el pie de página para poder rehacerlos luego.
\fancyhf{}
% Espacios en el documento:
\linespread{1.1}                        % Espacio entre líneas.
\setlength\parindent{0pt}               % Selecciona la indentación para cada inicio de párrafo.
% Cabecera del documento. Se ajusta la línea de la cabecera.
\renewcommand\headrule{
	\begin{minipage}{1\textwidth}
	    \hrule width \hsize
	\end{minipage}
}
% Texto de la cabecera:
\lhead{\subject}                          % Parte izquierda.
\chead{}                                    % Centro.
\rhead{\doctitle \ - \docsubtitle}              % Parte derecha.
% Pie de página del documento. Se ajusta la línea del pie de página.
\renewcommand\footrule{
\begin{minipage}{1\textwidth}
    \hrule width \hsize
\end{minipage}\par
}
\lfoot{}                                                 % Parte izquierda.
\cfoot{}                                                 % Centro.
\rfoot{Page\ \thepage\ of\ \protect\pageref{LastPage}} % Parte derecha.


%----------------------------------------------------------------------------------------
%   MATEMÁTICAS
%----------------------------------------------------------------------------------------

% Paquetes para matemáticas:
\usepackage{amsmath, amsthm, amssymb, amsfonts, amscd} % Teoremas, fuentes y símbolos.
\usepackage{dsfont} % new fonts added by contributors
\usepackage{tikz-cd} % para diagramas conmutativos
\usepackage[mathscr]{euscript}
\let\euscr\mathscr \let\mathscr\relax% just so we can load this and rsfs
\usepackage[scr]{rsfso}
\newcommand{\powerset}{\raisebox{.15\baselineskip}{\Large\ensuremath{\wp}}}
 % Nuevo estilo para definiciones
\usepackage{hyperref}

%-----------------------------------------------------------------------------------------------------
%	BIBLIOGRAFÍA
%-----------------------------------------------------------------------------------------------------

\usepackage[backend=bibtex, style=numeric]{biblatex}
\usepackage{csquotes}

\addbibresource{references.bib}

%-----------------------------------------------------------------------------------------------------
%	PORTADA
%-----------------------------------------------------------------------------------------------------
% Elija uno de los siguientes formatos.
% No olvide incluir los archivos .sty asociados en el directorio del documento.
%\usepackage{title1}
\usepackage{title2}
%\usepackage{title3}

%-----------------------------------------------------------------------------------------------------
%	TÍTULO, AUTOR Y OTROS DATOS DEL DOCUMENTO
%-----------------------------------------------------------------------------------------------------

% Título del documento.
\newcommand{\doctitle}{The Group Law for Edwards Curves}
% Subtítulo.
\newcommand{\docsubtitle}{}
% Fecha.
\newcommand{\docdate}{}
% Asignatura.
\newcommand{\subject}{}
% Autor.
\newcommand{\docauthor}{Rodrigo Raya Castellano}
\newcommand{\docaddress}{Technical University of Munich}
\newcommand{\docemail}{}

%-----------------------------------------------------------------------------------------------------
%	RESUMEN
%-----------------------------------------------------------------------------------------------------

% Resumen del documento. Va en la portada.
% Puedes también dejarlo vacío, en cuyo caso no aparece en la portada.
%\newcommand{\docabstract}{}
\newcommand{\docabstract}{}

\begin{document}

\makeatletter\renewcommand{\ALG@name}{Algoritmo}

\maketitle

%-----------------------------------------------------------------------------------------------------
%	ÍNDICE
%-----------------------------------------------------------------------------------------------------

% Profundidad del Índice:
%\setcounter{tocdepth}{1}

\newpage
\tableofcontents
\newpage

\section{Introduction}
The development of formal proofs of the group law of elliptic curves is very old. However, most of the works we are aware of focus on the group law for elliptic curves in Weierstrass form. Proofs varied largely on the theoretical complexity of the arguments used. For instance, in \cite{bartzia2014formal}, the authors develop a formal library for elliptic curves and the group law is shown establishing an isomorphism between the elliptic curve and its Picard group of divisors. Simpler arguments exist. For instance, \cite{thery2007proving} provides a more practical approach based on computationally intensive reasoning techniques rather than in sophisticated mathematical concepts. Here the author already points out one of the main challenges that elliptic curve verification presents to a modern theorem prover: 

\begin{displayquote}
To translate this 7 page long paper proof in a theorem prover was a real challenge. In fact, the proof relies on some non-trivial computations that the author advises to check using a computer algebra system such as CoCoA. The main difficulty has been to find an effective way to cope with these computations inside our proof system. 
\end{displayquote}

The author of this report conducted a similar case study two years ago \cite{techreport} following closely \cite{russinoff2017computationally}. This experience made him realise that the problem of verifying elliptic curves was not trivial at all and in fact, constitutes an ideal experiment to test the health of a theorem prover since it is both simple to understand and difficult to implement. Far are the times in which some textbooks could summarise the question as follows \cite{silverman1992rational}:

\begin{displayquote}
Of course, there are an awful lot of special cases to consider, such as when one of the points is the negative of the other or two of the points coincide. But in a few days you will be able to check associativity using these formulas. So we need say nothing more about the proof of the associative law!
\end{displayquote}

There is already an entry on the Archive of Formal Proofs that formalizes elliptic curves in Weierstrass form \cite{Elliptic_Curves_Group_Law-AFP}. Nevertheless, there is much less work done on elliptic curves in Edwards form. This can be explained by their relative novelty as they were first proposed in 2007 by Harold Edwards \cite{edwards2007normal}. 

Despite this, we find a similar situation to that encountered with Weierstrass curves. On the one hand, there exist proofs whose mathematical content is quite deep. For instance,  Edwards himself proposed an approach based on differential equations in connection with previous work by Euler and Gauss. Another approach is to build a so called rational equivalence between curves in Edwards form and curves in Weierstrass form \cite{das2008pairing} and then transfer the group properties from one to the other. A more practical approach has more recently been proposed and partially mechanized by Hales \cite{edwards2007normal}. In the present work, this last approach was followed.

\pagebreak
\section{Edwards curves}
\input{edwards/edwards}
\pagebreak
\section{Isabelle tools used}
\subsection{The use of locales}

During this project, I had a first contact with the tools that Isabelle uses to manage theories. These are not normally introduced in a basic course on the proof assistant and we had to explore on our own using the existing documentation \cite{ballarin2010tutorial}. We worked both with external locales (such as comm\_group):

\begin{figure}[!htbp]
	\centering
	\includegraphics[width=\linewidth,height=\textheight,keepaspectratio]{img/group_law.png}
	\caption{The group law theorem for affine elliptic curves using comm\_group locale}
	\label{fig:grouplaw}
\end{figure}

and our own locales that served to structure and particularize the theories:

\begin{figure}[!htbp]
	\centering
	\includegraphics[width=0.8\linewidth,height=0.8\textheight,keepaspectratio]{img/structure.png}
	\caption{General structure of the theory organized in locales}
	\label{fig:grouplaw}
\end{figure}
	
\subsection{The algebra method}

A good understanding of the proving tools at our disposal can save a lot of work. The algebra method \cite{wenzel2019isabelle} has as one of its basic functionalities solving the following quantified formula: 

\begin{align*}
\forall x_1 \ldots x_n. &  \\        
& e_1(x_1,\ldots,x_n) = 0 \land \ldots \land e_m(x_1,\ldots,x_n) = 0 \to \\
& (\exists y_1, \ldots, y_k. \\
& p_{11}(x_1,\ldots,x_n) y_1 + \ldots + p_{1k}(x_1,\ldots,x_n) y_k = 0 \land \\
& \ldots \land \\
& p_{t1}(x_1,\ldots,x_n) y_1 + \ldots + p_{tk}(x_1,\ldots,x_n) y_k = 0) \\
&
\end{align*}

where $e_1,\ldots,e_n$ and $p_{ij}$ are multivariate polynomials in the indicated variables. In our case, $e_i$ could be the hypothesis that we have certain point in the curve and the exists fragment corresponds to the search of division quotients. As a direct consequence, one does not always need to copy the polynomial quotients obtained with Mathematica. Here is an example:

\begin{figure}[!htbp]
	\centering
	\includegraphics[width=0.8\linewidth,height=0.8\textheight,keepaspectratio]{img/poly_expr.png}
	\caption{Example of polynomial division with the algebra method}
	\label{fig:groebner}
\end{figure}

The variables gxpoly\_expr, gypoly\_expr correspond to given polynomials that are computed explicitly with Mathematica. However, in our case, there is no need to copy the expressions for $r1,r2,r3$ since all we need is their existence. This is in contrast to the situation found in \cite{hales2016group}

\subsection{The use of Gröbner basis}

Hales explicitly uses Gröbner basis to go through the proof of dichotomy. This property is fundamental to establish the well-definition of the group law for projective Edwards curves. To be precise, it would be need in Isabelle whether a given set of polynomials corresponds to a Gröbner basis of a given polynomial ideal. However, the representation of this theory is quite different to ours:

\textit{Essentially, polynomials are represented as ordered lists of monomials, where monomials are represented as pairs consisting of a coefficient and a power-product (I.e., something like $x_0*y_0$); power-products are represented similarly, as ordered lists of indeterminate-exponent-pairs. Indeterminates, finally, are typically represented by natural numbers (although this can be changed easily). See also \cite{maletzky2018grobner}.}

While in practice the computation of Gröbner basis in Isabelle might look like in Figure \ref{fig:groebner}.

\begin{figure}[!htbp]
	\centering
	\includegraphics[width=0.8\linewidth,height=0.8\textheight,keepaspectratio]{img/groebner.png}
    \caption{Gröbner basis computation in Reduced\_GB\_Examples.thy}
    \label{fig:groebner}
\end{figure}

more expertise on this theory would be needed in order to connect it with our own. This can be considered as future work since having at our disposal an easy access to verified Gröbner basis computation would have speeded the proof process significantly. 

On the other hand, the algebra method is not capable to solve this problem either. While it computes Gröbner basis for the purpose of proving whether a polynomial is in the generated ideal, this computation is done outside the logic. In particular, it is not proved that the computed basis is indeed a Gröbner basis. 

As a last alternative, one could check by hand that the involved identities hold. This is what Hales suggests in the article when he says: \textit{In particular, our approach does not require the use of Gröbner bases (except in Lemma 4.3.2 where they make an easily avoidable appearance)}. 

Here is one example of such a hand computation. We follow the notation in \cite{hales2016group}. Namely, we focus on equation 19:

$$(x_0^2 - x_1^2,y_0^2 - x_1^2,x_0y_0 - x_1 y_1) \equiv (0,0,0) \; mod \; S_{\pm}$$

where $S_{\pm}$ is the Gröbner basis associated with certain polynomials known to evaluate to zero. One then deduces one by one the corresponding equations:

To start, one deduces the third equality:

$$\delta' = x_0 y_0 \delta_{0x} x_1 y_1 \Big(\frac{1}{t x_0}\Big) \Big(\frac{1}{t y_0}\Big) = 
x_0 y_0 \Big(1 - \frac{t^2 x_1 y_1}{t^2 x_0 y_0}\Big) = x_0y_0 - x_1 y_1$$

Since by assumption $\delta_{-} = 0$ we have the third equality and we note:

$$(1) \; x_0 = x_1 \Big(\frac{y_1}{y_0}\Big)$$

where we have that $x_0,y_0,x_1,y_1 \neq 0$.

For the second, we have:

$$\delta_{+} = t x_0 t_0 \delta_{1x} x_1 y_1 \Big(\frac{1}{tx_0}\Big) \Big(\frac{1}{ty_0}\Big) = t x_0 y_0 \Big(\frac{y_1}{t x_0} - \frac{x_1}{t y_0}\Big) = y_0 y_1 - x_0 x_1 \stackrel{1}{=} y_0 y_1 - x_1^2 \Big(\frac{y_1}{y_0}\Big) = \frac{y_1}{y_0} (y_0^2 - x_1^2)$$

since $\delta',\delta_{+} = 0$, we have the second equation. We also note: 

$$(2) \; \frac{y_1}{y_0} (y_0^2 - x_1^2) = 0$$

Finally, the third equation is obtained as follows:

$$x_0^2 - y_1^2 \stackrel{1}{=} x_1^2 \frac{y_1^2}{y_0^2} - y_1^2 = \Big(\frac{y_1^2}{y_0^2}\Big) (x_1^2 - y_0^2) \stackrel{2}{=} \frac{y_1^2}{y_0^2} (x_1^2 - y_0^2) \stackrel{2}{=} 0$$

One should note that not all the polynomials of $S_{\pm}$ were used in this deduction.

\subsection{The representation of projective elliptic curves}

We have discussed the representation of projective Edwards curves as described in \cite{hales2016group}. The challenge to represent this description is similar to the representation of equivalence classes in \cite{paulson2006defining}. Here is, to start with, the definition of projective addition:

\begin{figure}[!htbp]
	\centering
	\includegraphics[width=\linewidth,height=\textheight,keepaspectratio]{img/proj_add.png}
	\caption{Projective addition on points}
	\label{fig:proj_add}
\end{figure}

A posteriori, one notes that it would be more convenient to have a more balanced definition without the if-else construct. The version in figure \ref{proj_add_domain} would save some work. For instance, to show that we are in the second branch of the if, we would not need to show that the first branch has a false guard.

\begin{figure}[!htbp]
	\centering
	\includegraphics[width=\linewidth,height=\textheight,keepaspectratio]{img/proj_add_domain.png}
	\caption{Balanced version of the projective addition on points}
	\label{fig:proj_add_domain}
\end{figure}

Before introducing proper projective addition, we introduce how are projective points represented:

\begin{figure}[!htbp]
	\centering
	\includegraphics[width=\linewidth,height=\textheight,keepaspectratio]{img/e_proj.png}
	\caption{Projective points representation}
	\label{fig:e_proj}
\end{figure}

So points that satisfy the equation given by $e'$ an which are non-zero are identified with their inversions modulo $\tau$. The other conditions in the equivalence relation only ensure the reflexivity property. It is on these classes of points that projective addition should work. Figure \ref{fig:proj_add_class} shows the adaption of proj\_add to classes. 

proj\_add\_class first selects those pairs that lead to some result, then it actually computes the result and finally identifies equivalent points. Much later, when we prove covering and well-definition we show that indeed the last identification provides only one class. This is the class that is selected with the\_elem in proj\_addition.

\begin{figure}[!h]
	\centering
	\includegraphics[width=\linewidth,height=\textheight,keepaspectratio]{img/proj_add_class.png}
	\caption{Projective addition on classes of points}
	\label{fig:proj_add_class}
\end{figure}









\pagebreak
\section{Notes on the proof}
\input{explanation/explanation}
\pagebreak
\section{Conclusions}
\input{conclusion/conclusion}
\pagebreak


\printbibliography

\end{document}